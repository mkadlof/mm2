\addcontentsline{toc}{chapter}{Bibliografia}
\begin{thebibliography}{99}

\bibitem{unfolded}
A.~L.~Fink.
\newblock {\em Natively unfolded proteins,}
\newblock Curr Opin Struct Biol {\bf 15}, 35–41.
\\

Najczęściej cytowana w~branży dynamiki molekularnej pozycja:
\bibitem{understanding}
D.~Frenkel, B.~Smit.
\newblock {\em Understanding Molecular Simulation – From Algorithms to Applications,} volume~1~of {\em Computational Science Series.}
\newblock Academic Press, A~Division of Harcourt, Inc., 2nd edition, 2002.
\\

Świetne wprowadzenie do fizycznych podstaw dynamiki molekularnej (lepsze od Frenkela\&Smita, zdaniem autora tego skryptu):
\bibitem{zuckerman}
D. M. Zuckerman.
\newblock {\em Statistical Physics of Biomolecules: An Introduction,}
\newblock CRS PRESS, 2010.
\\

Praca, którą będziemy omawiać w~ramach II programu zaliczeniowego:
\bibitem{dill}
K. A. Dill.
\newblock {\em Principles of protein folding—a perspective from simple exact models,}
\newblock Protein Science, vol. 4, no. {\bf 4}, pp. 561–602, 1995.
\\

Strona kursu dynamiki molekularnej (dobry opis algorytmów termostatowych):
\bibitem{dobraStronka}
Cameron Abrams.
\newblock \texttt{www.pages.drexel.edu/\textasciitilde cfa22/msim/msim.html}
\newblock {\em Department of Chamical Engineering, Drexel University, Philadelphia.}

	
\end{thebibliography}
